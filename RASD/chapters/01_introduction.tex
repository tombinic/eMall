\chapter{Introduction}
\section{Purpose}
This document represents the \textbf{R}equirements \textbf{A}nalysis and \textbf{S}pecification \textbf{D}ocument (RASD). Its purpose is to present a thorough and meticulous description of the e-Mobility for All (eMall) system in terms of functional and non-functional requirements highlighting real possible use cases, system limits and constraints and finally end users interaction. The following document it is also a strong baseline for project planning and for a future software evaluation.\\

The audience of this document includes end users that are mostly interested in an high level description of the features and developers who have to implement the requirements presented in this document.
\subsection{Goals}
\label{sec:goals}
Nowadays, the urban and sub-urban mobility is one of the main causes of pollution which afflicts our planet, so to reduce the carbon footprint the usage of electric vehicles is encouraged as much as possible. The main goal of eMall is to help the spreading of this kind of transports focusing on the following general points:
\begin{table}[H]
\resizebox{\textwidth}{!}{%
\begin{tabular}{|c|l|}
\hline
\rowcolor[HTML]{b8c8d5}
\multicolumn{1}{|c|}{\cellcolor[HTML]{b8c8d5}\textbf{Goal}} & \multicolumn{1}{c|}{\cellcolor[HTML]{b8c8d5}\textbf{Description}} \\ \hline
G.1 \label{G.1}& \begin{tabular}[c]{@{}l@{}}Allow end users to find a most suitable charging station for them, according to \\ various criteria.\end{tabular} \\ \hline
G.2 \label{G.2}& Allow end users to make bookings and manage them. \\ \hline
G.3 \label{G.3}& Allow end users to start/monitor/stop the charging process. \\ \hline
G.4 \label{G.4}& Allow end users to automatically pay for a charge. \\ \hline
G.5 \label{G.5}& \begin{tabular}[c]{@{}l@{}}Allow the charging station administrators to monitor various information about their\\ charging stations as the internal and external status.\end{tabular} \\ \hline
G.6 \label{G.6}& \begin{tabular}[c]{@{}l@{}}Allow the charging station administrators to retrieve information about the price\\ from the energy providers also choosing which supplier to buy from.\end{tabular} \\ \hline
G.7 \label{G.7}& \begin{tabular}[c]{@{}l@{}}Allow the charging station administrators to decide how to use purchased energy\\ and also decide its selling price.\end{tabular} \\ \hline
\end{tabular}
}
\end{table}
\clearpage
\section{Scope}
Electric mobility for All (eMall) is a \textbf{user-friendly} application which is intended to facilitate the use of the system to electric vehicle owners and charging station administrators.\\
To reach these goals, the application provides an interface for end users that allows them \textbf{to search for a nearby charging station} that has a certain type of charging sockets (slow, fast, rapid), visualizing also its special offers. Consequently, end users \textbf{can charge their vehicles} in a specific station by reserving a time slot receiving a notification when the charge is completed. This feature allows to avoid traffic at the station and reduce the simultaneous presence of vehicles. The system also provides a full integrated payment system that enables end users to pay for the service.\\
The software also proposes a very smart feature: indeed, is able to proactively suggest the user to go and charge the vehicle, depending on the status of the
battery, his schedule, any special offers available, and the availability of charging slots at the identified stations.\\
The platform includes also a charging station management system which is used by the charging station administrators \textbf{to manage the infrastructure (batteries, etc) of charging stations} and, moreover, it has the possibility to buy the energy from $3^{rd}$ parties in the smartest possible way and distribute it to end user vehicles. The management system will also \textbf{keep track of the internal and external status} of each charging station respecting the physical constraints of the infrastructure.
\subsection{World phenomena}
\begin{table}[H]
\resizebox{\textwidth}{!}{%
\centering
\begin{tabular}{|c|l|}
\hline
\rowcolor[HTML]{b8c8d5} 
\multicolumn{1}{|c|}{\cellcolor[HTML]{b8c8d5}\textbf{World Phenomena}} & \multicolumn{1}{c|}{\cellcolor[HTML]{b8c8d5}\textbf{Description}} \\ \hline
WP.1 & The end user needs to charge his vehicle.               \\ \hline
WP.2 & The end user drives to the chosen charging station.     \\ \hline
WP.3 & The end user plugs in the cable in the designated charging socket. \\ \hline
WP.4 & The end user unplugs the charging cable.                \\ \hline
WP.5 & The end user leaves the charging station.              \\ \hline
WP.6 & \begin{tabular}[c]{@{}l@{}}The charging station administrator owns the charging station.\end{tabular}      \\ \hline
\end{tabular}
}
\end{table}

\subsection{Shared phenomena}
% Please add the following required packages to your document preamble:
% \usepackage[table,xcdraw]{xcolor}
% If you use beamer only pass "xcolor=table" option, i.e. \documentclass[xcolor=table]{beamer}
\begin{table}[H]
\resizebox{\textwidth}{!}{%
\centering
\begin{tabular}{|c|l|}
\hline
\rowcolor[HTML]{b8c8d5} 
\textbf{Shared Phenomena} & \multicolumn{1}{c|}{\cellcolor[HTML]{b8c8d5}\textbf{Description}} \\ \hline
SP.1 & \begin{tabular}[c]{@{}l@{}}The system shows charging stations nearby, their cost \\ and any special offer they have.\end{tabular} \\ \hline
SP.2 & The end user can book a time slot at the charging station. \\ \hline
SP.3 & The end user can start the charging process. \\ \hline
SP.4 & \begin{tabular}[c]{@{}l@{}}The end user can pay for the service, then he receives the invoice\\ through an email. \end{tabular}\\ \hline
SP.5 & \begin{tabular}[c]{@{}l@{}}The system suggests the most suitable charging station\\ for a certain end user.\end{tabular} \\ \hline
SP.6 & \begin{tabular}[c]{@{}l@{}}The charging station administrator acquires energy from energy providers\\ dinamically chosen.\end{tabular} \\ \hline
SP.7 & \begin{tabular}[c]{@{}l@{}}The charging station administrator or the system decide whether\\ or not to buy energy or use batteries.\end{tabular} \\ \hline
SP.8 & \begin{tabular}[c]{@{}l@{}}The system allows the charging station administrator to know\\ about the status of his charging stations.\end{tabular} \\ \hline
SP.9 & \begin{tabular}[c]{@{}l@{}}The interaction between the end user and the charging socket\\ happens through a QR-Code scan.\end{tabular} \\ \hline
\end{tabular}
}
\end{table}
\section{Definitions, Acronyms, Abbreviations}
\subsection{Definitions}
\begin{table}[H]
\resizebox{\textwidth}{!}{%
\begin{tabular}{|c|l|}
\hline
\rowcolor[HTML]{B8C8D5} 
\textbf{Term}    & \multicolumn{1}{c|}{\cellcolor[HTML]{B8C8D5}\textbf{Definition}}        \\ \hline
End user        & \begin{tabular}[c]{@{}l@{}}Identifies electric vehicles owners who use the service, \\ can also be referred as user.\end{tabular}                \\ \hline
Charging station & Place of charging of electric vehicles.                                 \\ \hline
Charging Station available &\begin{tabular}[c]{@{}l@{}} At the moment at least one socket is free.\end{tabular} \\ \hline
Charging Station not available &\begin{tabular}[c]{@{}l@{}} At the moment all the sockets are occupied.\end{tabular} \\ \hline
Charging socket  & Single charging point.                                                  \\ \hline
Booking   &  \begin{tabular}[c]{@{}l@{}}Reserved time slot in a specific station for \\charging vehicles. It also can be referred to as Reservation.\end{tabular}     \\ \hline
Time slot        & Slot of time in which customers can charge their vehicles.               \\
\hline
QR-Code &  \begin{tabular}[c]{@{}l@{}} A QR-code is a type of matrix barcode (or two-dimensional barcode). \end{tabular}  \\ \hline
System &
  \begin{tabular}[c]{@{}l@{}}Set of hardware and software tools that provide the \\ desired service. It can be considered as eMall,\\ Application and Platform.\end{tabular} \\ \hline
  Internal status &
  \begin{tabular}[c]{@{}l@{}}For internal status of a charging station: the \\ amount of energy available in its batteries, the number \\ of vehicles being charged and, for each charging vehicle, \\the amount of power absorbed and time left to the end\\ of the charge. \end{tabular} \\ \hline
    External status &\begin{tabular}[c]{@{}l@{}}For external status of a charging station:\\the number of charging sockets available, their type \\ such as slow/fast/rapid, their cost, and, if all sockets\\ of a certain type are occupied, the estimated amount \\of time until the first socket of that type is freed. \end{tabular} \\ \hline
\end{tabular}
}
\end{table}

\subsection{Acronyms}

\begin{table}[H]
\centering
\begin{tabular}{|c|l|}
\hline
\rowcolor[HTML]{B8C8D5} 
\textbf{Acronyms} & \multicolumn{1}{c|}{\cellcolor[HTML]{B8C8D5}\textbf{Meaning}} \\ \hline
eMall             & e-Mobility for All                                            \\ \hline
eMSP              & e-Mobility Service Provider                                   \\ \hline
CPO               & Charging Point Operator                                       \\ \hline
CPMS              & Charging Point Management System                              \\ \hline
DSO               & Distribution System Operator                                  \\ \hline
API               & Application Programming Interface                             \\ \hline
GPS               & Global Positioning System                                     \\ \hline
PC                & Personal Computer                                             \\ \hline
HTTPS             & Hyper Text Transfer Protocol Secure                           \\ \hline
HTTP          & Hyper Text Transfer Protocol                           \\ \hline
WiFi              & Wireless Fidelity                                             \\ \hline
LTE                & Long Term Evolution \\ \hline
3G                & Third-Generation Wireless                                     \\ \hline
4G                & Fourth-Generation Wireless                                    \\ \hline
5G                & Fifth-Generation Wireless                                     \\ \hline
SSL               & Secure Socket Layer                                           \\ \hline
SHA-256           & Secure Hash Algorithm                                          \\ \hline
SCA               & Strong Customer Authentication                                  \\ \hline
a.m.               & Ante Meridiem                                                  \\ \hline
p.m.               & Post Meridiem                                                  \\ \hline
\end{tabular}

\end{table}
\subsection{Abbreviations}

\begin{table}[H]
\centering
\begin{tabular}{|c|l|}
\hline
\rowcolor[HTML]{B8C8D5} 
\textbf{Abbreviations} & \multicolumn{1}{c|}{\cellcolor[HTML]{B8C8D5}\textbf{Meaning}} \\ \hline
WP & World Phenomena  \\ \hline
SP & Shared Phenomena \\ \hline
G  & Goal             \\ \hline
R  & Requirement             \\ \hline
NFR  & Non Functional Requirement             \\ \hline
D  & Domain Assumption             \\ \hline
w.r.t. & with reference to \\ \hline
e.g. & exempli gratia \\ \hline
i.e. & id est \\ \hline
etc. & etcetera \\ \hline
\end{tabular}
\end{table}
\subsection{Important terminology}
\begin{itemize}
    \item The \textbf{CPO} is the charging station’s administrator and owner (for example an e-mobility company).
    \item The \textbf{eMSP} is the system module which interfaces with the end users, it includes services and a mobile application.
    \item The \textbf{CPMS} is the system module which interfaces with the CPOs, and manages the charging stations.
\end{itemize}
The eMall platform provides both a CPMS and an eMSP modules.
%\section{Revision History}
\section{Reference Documents}
\begin{itemize}
    \item \emph{Course slides on WeeBeep.}
    \item \emph{RASD assignament document.}
    \item \emph{RASD review by Prof. M. Camilli.}
\end{itemize}

\section{Document Structure}
The structure of this RASD document is the following:
\begin{enumerate}
    \item \textbf{Introduction}: In this section is presented the purpose of this document highlighting in particular the main goals, the audience which is referred to, the identification of the product and application domain  describing world and shared phenomena and, lastly, the terms definitions.
    \item \textbf{Overall Description}: This chapter describes the possible scenarios of the platform, the shared phenomena presented at the beginning of the document and assumptions on the domain of the application.
    \item \textbf{Specific Requirements}: Includes all the requirements in a more specific way than the "Overall Description" section. Moreover, it is useful to show functional requirements in terms of use cases diagrams, sequence/activity diagrams.
    \item \textbf{Formal Analysis Using Alloy}: Includes Alloy models which are used for the description of the application domain and his properties, referring to the operations which the system has to provide.
    \item \textbf{Effort Spent}: This section shows the effort spent in terms of time for each team member and the whole team.
    \item \textbf{References}: Includes all documents that were helpful in drafting the RASD.
\end{enumerate}
