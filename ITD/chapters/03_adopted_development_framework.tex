\chapter{Adopted Development Framework}
Below, are described firstly the frameworks used and then the most import modules of the three software implemented.
\section{Frameworks}
\begin{itemize}
    \item \textbf{React}: React.js is a JavaScript library for building user interfaces. It was developed by Facebook and is now maintained by Meta and a community of individual developers and companies. React allows developers to build large, complex user interfaces by breaking them down into smaller, reusable components. Each component can manage its own state, making it easier to build and maintain large applications. React uses a virtual DOM (Document Object Model) to optimize updates and rendering, making it fast and efficient. React can be used with a variety of programming languages and tools, making it a versatile choice for building web applications. It is widely used for building single-page applications, mobile apps, and progressive web apps.
    \item \textbf{NodeJS}: NodeJS is an open-source, cross-platform JavaScript runtime environment that executes JavaScript code outside of a web browser. It is commonly used for building server-side applications.
    \item \textbf{Flutter}: Flutter is an open-source mobile application development framework created by Google. It uses the Dart programming language and offers a fast, expressive, and flexible way to build natively compiled applications for mobile, web, and desktop from a single codebase. With its rich set of customizable widgets and tools, Flutter makes it easy for developers to create high-performance and visually appealing apps that run smoothly on both Android and iOS platforms.
    \end{itemize}
    \section{Modules}
    \begin{itemize}
        \item \textbf{Chai}: Chai is a JavaScript BDD/TDD assertion library for node and the browser that can be paired with any testing framework. It provides a clean and simple interface for making assertions and can be used for both unit and integration tests. Chai provides a large number of assertions for different types of values and objects, making it a versatile and comprehensive tool for testing.
        \item \textbf{Mocha}: Mocha is a JavaScript test framework running on Node.js. It provides a simple and flexible API for writing and executing tests for Node.js applications. Mocha tests can be run in series or parallel, and it supports asynchronous testing, which allows for testing code with async functions. Mocha also offers features such as reporting, test retries, and test hooks.
        \item \textbf{Socket.io}: Socket.IO is a JavaScript library for real-time web applications. It enables real-time, bidirectional communication between web clients and servers. With Socket.IO you can emit and receive events in real-time, allowing for the creation of real-time applications such as chat rooms, online games, and live-updating pages. The library abstracts the low-level details of WebSockets and provides a simple, event-driven API for working with real-time data. Both the Socket.io server and client were used for development.
        \item\textbf{MUI}: Material-UI is a popular React UI library that implements Google's Material Design specifications. It provides a set of components that follow Material Design guidelines and can be used to build modern and responsive user interfaces. The library provides a wide range of pre-built components for various UI elements such as buttons, cards, icons, forms, and more. The components are highly customizable, allowing developers to style them to match the look and feel of their application. Material-UI also provides support for theming, making it easy to change the color palette and other visual styles across an entire application. Overall, Material-UI is a useful library for developers looking to build beautiful and functional user interfaces with React. 
        \item\textbf{Axios}: Axios is a popular JavaScript library that enables communication with HTTP servers, making it easy to retrieve and send data over the internet. It provides a simple, promise-based API for making HTTP requests, including GET, POST, PUT, DELETE, and others. Axios supports all modern browsers and is designed to work with both Node.js and web browsers. Axios provides several features that make it a popular choice for making HTTP requests, including automatic transforming of JSON data, support for canceling requests, and an ability to request data from multiple sources. Axios is often used for building web and mobile applications, particularly with JavaScript frameworks such as React.
        \item\textbf{CryptoJS}: CryptoJS is a library of cryptographic algorithms written in JavaScript. It provides a wide range of cryptographic functions, including AES encryption, SHA hashes, HMAC, Base64 encoding and decoding, and more. The library can be used in web browsers as well as with Node.js. CryptoJS is designed to provide simple, high-level APIs for developers, abstracting away the underlying cryptographic details. It is a widely-used library for adding security to web applications, and is especially useful for implementing encryption and decryption in the browser, where direct access to the operating system's cryptographic libraries is not available. Overall, CryptoJS is a valuable tool for developers looking to add cryptography to their web applications.
        \item\textbf{OpenGeocoder}: OpenGeocoder is a Node.js library that provides geocoding and reverse geocoding functionalities. Geocoding is the process of converting an address or place name into coordinates, while reverse geocoding is the process of converting coordinates into a human-readable address. OpenGeocoder can be used to convert addresses into latitude and longitude coordinates and vice versa, making it useful for mapping applications and location-based services.
        \item \textbf{Google\_maps\_flutter}: Google\_maps\_flutter is a Flutter library for integrating Google Maps in mobile applications. It provides a high-level API for displaying maps and markers, allowing developers to easily add Google Maps functionality to their Flutter applications. The library provides a number of customization options, including the ability to change the map's style, add markers, and handle map gestures. With Google\_maps\_flutter, developers can create engaging and interactive maps experiences with just a few lines of code.
        \item \textbf{http}: The "http" library in Flutter is a Dart package for making HTTP requests. It allows developers to send and receive data from web servers over the internet. The library provides a simple and convenient API for sending HTTP requests, including support for both GET and POST methods. Additionally, the library can handle common HTTP responses, such as redirects and error codes, making it easy to add network communication to a Flutter app. With the "http" library, developers can quickly and easily access RESTful APIs and other web services, making it a valuable tool for building robust and connected mobile applications.
        \item \textbf{Pretty\_qr\_code}: pretty\_qr\_code is a Flutter library for generating and displaying visually appealing QR codes. It offers a number of customization options for changing the appearance of the QR code, including the ability to set the color, add a logo, and adjust the border size. This library provides an easy-to-use and flexible API for generating QR codes, allowing developers to quickly and easily add QR code functionality to their Flutter applications. With pretty\_qr\_code, developers can create eye-catching QR codes that stand out and are easy to scan, making it a useful tool for implementing QR code-based features in mobile apps.
        \item\textbf{ZXing}: ZXing is an open-source, multi-format 1D/2D barcode image processing library implemented in Java. It is widely used for reading and generating QR codes, as well as other barcode formats such as UPC-A, EAN-8, and more. The library has been ported to several other programming languages, including JavaScript, allowing it to be used in web applications. ZXing provides a simple and efficient API for reading barcodes from images, making it a popular choice for developers building barcode scanning applications. The library can be integrated with a variety of platforms, including Android, iOS, and the web, and is well-documented, making it easy to use and customize.
\end{itemize}
\clearpage
\section{Programming languages}
\subsection{JavaScript}
\textbf{JavaScript} is a high-level, dynamic, and interpreted programming language. It is widely used for creating interactive web applications and front-end development. Some of its advantages include:
\begin{itemize}
    \item \textbf{Easy to learn}: JavaScript has a simple syntax and is easy to learn for developers who have prior programming experience.
    \item \textbf{Versatility}: JavaScript can be used for a wide range of tasks, including server-side programming, desktop application development, and mobile app development.
    \item \textbf{Popularity}: JavaScript is one of the most popular programming languages and has a large community of developers who create and share useful resources.
    \item \textbf{Interactivity}: JavaScript allows for the creation of interactive elements on websites, such as animations, pop-up windows, and more.
\end{itemize}
However, JavaScript also has some disadvantages, including:
\begin{itemize}
    \item \textbf{Lack of security}: JavaScript is often used for client-side scripting, which makes it vulnerable to security threats such as cross-site scripting attacks.
    \item \textbf{Browser compatibility issues}: Different browsers have different levels of support for JavaScript, which can lead to compatibility issues.
    \item \textbf{Performance limitations}: JavaScript can become slow when used for intensive tasks or when working with large amounts of data.
\end{itemize}
In conclusion, JavaScript is a popular and versatile programming language that is widely used for front-end development. While it has its disadvantages, its benefits make it an attractive choice for many developers and projects.
\subsection{Dart}
\textbf{Dart} is an open-source, statically typed programming language developed by Google. It is used for building web, server, and mobile applications. Some of the benefits of using Dart include:
\begin{itemize}
    \item \textbf{Easy to learn}: Dart has a syntax that is easy to understand for developers who have experience with other programming languages.
    \item \textbf{Scalability}: Dart is designed for large-scale applications and has features that support scalability and maintainability.
    \item \textbf{Fast performance}: Dart compiles to native code, which results in fast performance and reduced latency compared to other interpreted languages.
    \item \textbf{Cross-platform development}: Dart can be used to build applications for the web, mobile devices, and servers, making it a good choice for cross-platform development.
\end{itemize}
However, Dart also has some drawbacks, including:
\begin{itemize}
    \item \textbf{Limited adoption}: Dart is not as widely used as other programming languages and has a smaller community of developers.
    \item \textbf{Steep learning curve}: While Dart's syntax is easy to understand, its features and structure can be difficult for new developers to grasp.
    \item \textbf{Lack of support for some libraries}: Dart's relatively limited popularity has resulted in limited support for certain libraries and frameworks.
\end{itemize}
In conclusion, Dart is a powerful language that is well-suited for building large-scale applications and for cross-platform development. While it has its limitations, its benefits make it a good choice for some projects and developers.
\subsection{Java (for mobile development)}
\textbf{Java} is a popular programming language that is widely used for mobile app development. It is an object-oriented language known for its robustness, security, and portability, making it well-suited for developing complex mobile applications that run on a variety of platforms. Java is the primary language for developing Android apps, and it can also be used to build cross-platform apps that run on multiple mobile operating systems. With its rich ecosystem of libraries and tools, Java provides developers with a wide range of options for building mobile apps. Whether you're building an Android app or creating a cross-platform mobile app, Java is a versatile and powerful language that can help you bring your ideas to life.
Advantages of using Java for mobile development:
\begin{itemize}
    \item \textbf{Cross-platform compatibility}: Java is platform-independent and can run on multiple platforms, making it a great choice for cross-platform mobile development.

    \item \textbf{Large development community}: Java has a large and active development community, which means that developers can access a wealth of resources, such as libraries, tools, and support forums.

    \item \textbf{Robust libraries}: Java has a wide range of libraries and frameworks for mobile development, making it easier to add functionality and speed up development time.

    \item \textbf{Security}: Java is known for its security features, making it a good choice for developing applications that handle sensitive data.

    \item \textbf{Performance}: Java is known for its performance, making it suitable for demanding mobile applications, such as games and augmented reality applications.
\end{itemize}
Disadvantages of using Java for mobile development:
\begin{itemize}
    \item \textbf{Resource-intensive}: Java applications can be resource-intensive, which can lead to slow performance on older and lower-end mobile devices.

    \item \textbf{High memory usage}: Java applications can use a lot of memory, which can be a problem on mobile devices with limited memory.

    \item \textbf{Fragmentation}: Java can suffer from fragmentation, with different versions of the platform running on different devices, which can lead to compatibility issues.

    \item \textbf{Debugging and maintenance}: Debugging and maintaining Java code can be more difficult than with other mobile development platforms, due to its complexity and verbosity.
\end{itemize}

\subsection{SQL}
\textbf{SQL} (Structured Query Language) is a standard programming language used for managing and manipulating data stored in relational databases. It is used to insert, update, and retrieve data from a database, as well as to create, modify, and delete database tables and relationships between them. SQL is widely used in enterprise applications, web applications, and data analysis, making it an essential tool for many organizations and developers. With its powerful query and data manipulation capabilities, SQL enables efficient and effective management of large amounts of structured data.\\
Advantages of using SQL:
\begin{itemize}
    \item \textbf{Widely used and supported}: SQL is a widely used and well-established standard, and is supported by a wide range of relational database management systems (RDBMS), including MySQL, Microsoft SQL Server, Oracle, and others.

     \item \textbf{Efficient data management}: SQL provides a powerful and flexible means of managing large amounts of structured data, with efficient querying, insertion, updating, and deletion capabilities.

     \item \textbf{High-level abstraction}: SQL provides a high-level abstraction of the underlying data, making it easier to work with data without having to understand the underlying physical storage structures.

     \item \textbf{Strong data integrity and security}: SQL provides strong data integrity and security features, such as constraints, transactions, and access controls, making it a good choice for applications that handle sensitive data.

     \item \textbf{Good scalability}: SQL databases can be scaled to handle large amounts of data and users, making them a good choice for applications with a high volume of data or users.
\end{itemize}

Disadvantages of using SQL:
\begin{itemize}
    \item \textbf{Limited scalability for certain use cases}: While SQL is generally good at scaling, it can be limited for certain use cases, such as big data and real-time analytics.

    \item \textbf{Rigid data structure}: SQL requires a rigid data structure, which can be inflexible and make it difficult to handle changing requirements.

    \item \textbf{Performance overhead}: SQL can have performance overhead, particularly for complex queries or large amounts of data.

    \item \textbf{Limited expressiveness}: SQL is limited in terms of expressiveness, compared to general-purpose programming languages, making it less suitable for certain use cases.
\end{itemize}




